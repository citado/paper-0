\قسمت{تعریف مساله}

در حوزه‌ی شبکه‌های \متن‌لاتین{LoRa} پژوهش‌های گسترده‌ای به ارزیابی اجزای مختلف شبکه پرداخته‌اند، اما هنوز
مساله ارتباط این اجزا و تاثیر آن‌ها بر یکدیگر در یک سیستم انتها به انتها نیاز به ارزیابی دارد.
همانطور که در پژوهش \مرجع{FernandesCarvalho2019} شبکه‌های \متن‌لاتین{LoRaWAN} بیشتر پیچیدگی را
در \متن‌لاتین{Backend} قرار داده‌اند بنابراین در کنار شبکه دسترسی، شبکه‌ی هسته نیز تاثیر زیادی در کارایی دارد.
از سوی دیگر تعریف باز ارائه شد توسط استاندارد، اجازه پیاده‌سازی‌های گوناگونی را می‌دهد که مقایسه آن‌ها را مساله‌ی مهمی می‌کند
\مرجع{FernandesCarvalho2019}.

\زیرقسمت{تعریف مساله}

در این پژوهش قصد داریم به ارزیابی کارایی شبکه هسته پشته پروتکلی \متن‌لاتین{LoRaWAN}
بپردازیم. در این ارزیابی دو پارامتر تاخیر و نرخ از دست رفت بسته‌ها را مدنظر داریم که دو پارامتر مهم
و تاثیرگذار بر تجربه کاربر هستند.
در مرحله اول با تعدادی شبیه‌سازی و محک‌های گوناگون تاثیر پارامترهای طراحی مختلف بر شبکه \متن‌لاتین{LoRaWAN}
را مرور می‌کنیم.
در ادامه بر اساس این ارزیابی‌ها و تئوری صف حالت تعادل را در یک سیستم \متن‌لاتین{LoRaWAN} شناسایی می‌کنیم.
این حالت تعادل می‌تواند برای استقرار مناسب سیستم و منابع موردنیاز استفاده شود.

پارامترهای کارایی در این پژوهش شامل نرخ دریافت بسته، میانگین تاخیر انتها به انتها و ۹۰ درصد بالای تاخیر انتها به انتها
هستند.

نرخ دریافت بسته عبارت از نسبت بسته‌های دریافت شده از سرور اپلیکیشن \متن‌لاتین{LoRaWAN}
نسبت به بسته‌های ارسال شده توسط \متن‌لاتین{Gateway}:

\[
  DR = \frac{N_{r}}{N_{t}}
\]

تاخیر انتها به انتها مدت زمانی است که طول می‌کشد بسته از \متن‌لاتین{Gateway} ارسال شده
و از طریق سرور اپلیکیشن دریافت شود.
